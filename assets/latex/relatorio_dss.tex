\documentclass[11pt,a4paper]{article}

% --- BLOCO UNIVERSAL DE PREAMBULO (Compativel com pdflatex basico) ---
\usepackage[a4paper, top=2.5cm, bottom=2.5cm, left=2cm, right=2cm]{geometry}
\usepackage[T1]{fontenc}
\usepackage[portuguese]{babel}

% Pacotes Adicionais
\usepackage{amsmath}
\usepackage{enumitem}
\usepackage{booktabs}
\usepackage{float}
\usepackage{listings}
\usepackage{xcolor}
\usepackage{tikz}
\usetikzlibrary{shapes,arrows,positioning}
\usepackage{hyperref}
\usepackage{indentfirst} % Garante indentacao no primeiro paragrafo das secoes

% Configuracao de Paragrafos
\setlength{\parindent}{1.5em}
\setlength{\parskip}{0.5em}

% Configuracao de Listagens de Codigo
\lstset{
    basicstyle=\tiny\ttfamily,
    breaklines=true,
    frame=single,
    backgroundcolor=\color{gray!10},
    keywordstyle=\color{blue},
    commentstyle=\color{green!50!black},
    stringstyle=\color{orange}
}

\begin{document}

% ==== PRIMEIRA PAGINA: CAPA PRINCIPAL ====
\begin{titlepage}
    \begin{center}
        \vspace*{1.5cm}
        {\huge \textbf{Universidade do Minho}} \\
        {\large Escola de Engenharia | Departamento de Inform\'atica}\\[1cm]
        {\Large \textbf{Desenvolvimento de Sistemas de Software}}\\[2cm]
        
        {\huge \textit{Sistema de Gest\~ao de Restaurantes Fast-Food}}\\[1cm]
        {\Large \textbf{Projeto Pr\'atico - Grupo TP-28}}\\[3cm]
        
        \textbf{URL do Reposit\'orio:} \\
        \url{https://github.com/marcosevegrand/DSS2526}
        
        \vfill
        {\large Braga, Janeiro de 2026}
    \end{center}
\end{titlepage}

% ==== SEGUNDA PAGINA: IDENTIFICACAO DOS ALUNOS ====
\begin{titlepage}
    \begin{center}
        \vspace*{1cm}
        {\Large \textbf{Identifica\c{c}\~ao dos Elementos do Grupo}}\\[2cm]
        
        \begin{tabular}{cc}
            \framebox{\parbox{4cm}{\centering \vspace{2cm} \textbf{Foto} \\ \small Francisco \vspace{2cm}}} & 
            \framebox{\parbox{4cm}{\centering \vspace{2cm} \textbf{Foto} \\ \small Hugo \vspace{2cm}}} \\
            \textbf{Francisco Martins} & \textbf{Hugo Soares} \\
            A106902 & A107293 \\[1.5cm]
            
            \framebox{\parbox{4cm}{\centering \vspace{2cm} \textbf{Foto} \\ \small Marco \vspace{2cm}}} & 
            \framebox{\parbox{4cm}{\centering \vspace{2cm} \textbf{Foto} \\ \small Nuno \vspace{2cm}}} \\
            \textbf{Marco Sevegrand} & \textbf{Nuno Rebelo} \\
            A106807 & A107372 \\
        \end{tabular}
        
        \vfill
        \textit{Documento de apoio \`a an\'alise do sistema desenvolvido.}
    \end{center}
\end{titlepage}

\tableofcontents
\newpage

\section{Descri\c{c}\~ao dos Resultados Obtidos}

O sistema desenvolvido para a gest\~ao de restaurantes permite a coordena\c{c}\~ao centralizada de uma cadeia de fast-food, ligando o momento da escolha do cliente ao processo de confec\c{c}\~ao na cozinha. Atrav\'es de uma an\'alise cuidada das necessidades operacionais, estrutur\'amos a aplica\c{c}\~ao para garantir que o fluxo de informa\c{c}\~ao entre o balc\~ao e a produ\c{c}\~ao seja cont\'inuo e organizado.

Um dos resultados mais significativos foi a separa\c{c}\~ao do sistema em tr\^es subsistemas distintos, mas complementares, cada um com a sua pr\'opria fachada especializada. O subsistema de venda (\textit{VendaFacade}) gere a intera\c{c}\~ao direta com o cliente, permitindo a personaliza\c{c}\~ao de produtos. O subsistema de produ\c{c}\~ao (\textit{ProducaoFacade}) organiza as tarefas nas esta\c{c}\~oes de trabalho, calculando tempos de espera reais. Por fim, o subsistema de gest\~ao (\textit{GestaoFacade}) oferece as ferramentas administrativas para o controlo de pessoal e stocks.

Do ponto de vista arquitetural, o projeto destaca-se pelo isolamento das responsabilidades. A l\'ogica de persist\^encia, baseada em SQL, est\'a completamente separada da l\'ogica de neg\'ocio atrav\'es de objetos de acesso a dados (DAOs). Cada fachada \'e independente e utiliza apenas os DAOs necess\'arios para o seu subsistema, eliminando depend\^encias desnecess\'arias. Esta decis\~ao permite que o sistema lide com volumes maiores de informa\c{c}\~ao sem perder desempenho e facilita futuras atualiza\c{c}\~oes na estrutura de dados ou na interface de utilizador.

\section{An\'alise de Requisitos}

A an\'alise focou-se na identifica\c{c}\~ao clara dos processos de neg\'ocio e na forma como as entidades se relacionam no contexto de v\'arios restaurantes.

\subsection{Modela\c{c}\~ao de Dom\'inio}
O modelo de dom\'inio identifica conceitos centrais como \textit{Pedido}, \textit{Estacao}, \textit{Funcionario} e \textit{Ingrediente}. As rela\c{c}\~oes estabelecidas permitem que o sistema saiba, por exemplo, quais os ingredientes necess\'arios para cada \textit{Produto} e em que \textit{Estacao} devem ser processados.

\begin{figure}[H]
    \centering
    \framebox{\parbox{0.8\textwidth}{\centering \vspace{3cm} \textbf{Diagrama de Modela\c{c}\~ao de Dom\'inio} \\ \small assets/diagrams/modelo\_de\_dominio.png \vspace{3cm}}}
    \caption{Representa\c{c}\~ao das entidades e rela\c{c}\~oes do dom\'inio.}
\end{figure}

\subsection{Diagramas de Casos de Uso}
Estes diagramas ilustram as intera\c{c}\~oes entre os atores (Cliente, Cozinheiro, Gestor) e o sistema, detalhando funcionalidades como a finaliza\c{c}\~ao de pagamentos ou o envio de mensagens administrativas.

\begin{figure}[H]
    \centering
    \framebox{\parbox{0.8\textwidth}{\centering \vspace{3cm} \textbf{Diagrama de Casos de Uso} \\ \small assets/diagrams/diagrama\_use\_cases.png \vspace{3cm}}}
    \caption{Casos de uso principais do sistema.}
\end{figure}

\section{Modela\c{c}\~ao Conceptual}

A transi\c{c}\~ao da an\'alise para a solu\c{c}\~ao t\'ecnica baseou-se na defini\c{c}\~ao de como as classes colaboram entre si para cumprir as regras de neg\'ocio.

\subsection{Diagramas de Classe e de Sequ\^encia}
A modela\c{c}\~ao conceptual define a estrutura das fachadas especializadas (\textit{VendaFacade}, \textit{ProducaoFacade}, \textit{GestaoFacade}) e como estas utilizam as entidades de dom\'inio e os DAOs para processar pedidos. O foco recai sobre a coordena\c{c}\~ao entre os controladores de interface e a l\'ogica centralizada em cada servi\c{c}o.

\begin{figure}[H]
    \centering
    \framebox{\parbox{0.8\textwidth}{\centering \vspace{2.5cm} \textbf{Diagramas de Classe (Conceptual)} \\ \small (Placeholder) \vspace{2.5cm}}}
    \quad
    \framebox{\parbox{0.8\textwidth}{\centering \vspace{2.5cm} \textbf{Diagramas de Sequ\^encia (Conceptual)} \\ \small (Placeholder) \vspace{2.5cm}}}
    \caption{Estrutura conceptual da solu\c{c}\~ao proposta.}
\end{figure}

\section{Solu\c{c}\~ao Implementada}

A implementa\c{c}\~ao reflete as decis\~oes tomadas durante o planeamento, utilizando uma arquitetura em camadas bem definida.

\subsection{Arquitetura, Componentes e Pacotes}
O c\'odigo est\'a organizado em pacotes que refletem a sua fun\c{c}\~ao:

\begin{itemize}
    \item \textit{dss2526.ui} --- Apresenta\c{c}\~ao (interfaces de utilizador para cada subsistema)
    \item \textit{dss2526.service} --- L\'ogica de neg\'ocio, dividida em tr\^es sub-pacotes:
        \begin{itemize}
            \item \textit{dss2526.service.venda} --- VendaFacade e opera\c{c}\~oes de venda
            \item \textit{dss2526.service.producao} --- ProducaoFacade e opera\c{c}\~oes de produ\c{c}\~ao
            \item \textit{dss2526.service.gestao} --- GestaoFacade e opera\c{c}\~oes de gest\~ao
        \end{itemize}
    \item \textit{dss2526.data} --- Persist\^encia de dados atrav\'es de DAOs e implementa\c{c}\~oes (SQL)
    \item \textit{dss2526.domain} --- Entidades de dom\'inio e abstra\c{c}\~oes
    \item \textit{dss2526.app} --- Configura\c{c}\~ao e inicializa\c{c}\~ao da aplica\c{c}\~ao
\end{itemize}

Cada fachada especializada implementa a sua interface correspondente (\textit{IVendaFacade}, \textit{IProducaoFacade}, \textit{IGestaoFacade}), oferecendo um contrato claro das opera\c{c}\~oes dispon\'iveis. As fachadas utilizam o padr\~ao Singleton para garantir uma \'unica inst\^ancia durante a execu\c{c}\~ao da aplica\c{c}\~ao.

A persist\^encia de dados \'e gerida por DAOs especializados (IngredienteDAO, RestauranteDAO, ProdutoDAO, PedidoDAO, etc.), que abstraem os detalhes de acesso \`a base de dados SQL. Cada fachada utiliza apenas os DAOs necess\'arios para o seu subsistema, promovendo baixo acoplamento.

\begin{figure}[H]
    \centering
    \framebox{\parbox{0.45\textwidth}{\centering \vspace{2cm} \textbf{Packages e Componentes} \\ \small (Placeholder) \vspace{2cm}}}
    \hfill
    \framebox{\parbox{0.45\textwidth}{\centering \vspace{2cm} \textbf{Diagramas de Sequ\^encia Reais} \\ \small (Placeholder) \vspace{2cm}}}
    \caption{Detalhamento da solu\c{c}\~ao efetivamente implementada.}
\end{figure}

\section{Manual de Utiliza\c{c}\~ao}

O manual descreve o percurso t\'ipico de cada utilizador dentro do sistema.

\subsection{Subsistema de Venda (Terminal do Cliente)}
Ao aceder ao terminal, o cliente seleciona o seu modo de consumo. Ap\'os escolher os produtos e realizar eventuais personaliza\c{c}\~oes, o sistema solicita o pagamento e emite um identificador de pedido, acompanhado do tempo de espera previsto.

A \textit{VendaFacade} coordena:
\begin{itemize}
    \item Listagem filtrada do cat\'alogo (com exclus\~oes por alerg\'enicos)
    \item Adi\c{c}\~ao e remo\c{c}\~ao de itens do pedido
    \item Processamento de pagamento (terminal ou caixa)
    \item C\'alculo de estimativas de espera baseado na dura\c{c}\~ao dos passos
\end{itemize}

\subsection{Subsistema de Produ\c{c}\~ao (Cozinha)}
O funcion\'ario escolhe o restaurante e a sua esta\c{c}\~ao espec\'ifica. O sistema apresenta as tarefas a realizar. Ao concluir um item, o estado \'e atualizado em tempo real, permitindo que o ecr\~a de entregas informe o cliente quando o pedido est\'a completo.

A \textit{ProducaoFacade} gerencia:
\begin{itemize}
    \item Distribui\c{c}\~ao de tarefas por esta\c{c}\~ao
    \item Atualiza\c{c}\~ao de estados de produ\c{c}\~ao
    \item C\'alculo de tempos reais de processamento
    \item Sincroniza\c{c}\~ao com o subsistema de venda
\end{itemize}

\subsection{Subsistema de Gest\~ao (Administra\c{c}\~ao)}
O gestor pode monitorizar o invent\'ario atrav\'es da \textit{LinhaStock}, enviar mensagens para os funcion\'arios na cozinha e gerir a configura\c{c}\~ao das unidades. Esta \'area centraliza as decis\~oes que afetam o funcionamento global do restaurante.

A \textit{GestaoFacade} oferece:
\begin{itemize}
    \item Controlo de stocks e ingredientes
    \item Gest\~ao de funcion\'arios e permiss\~oes
    \item Configura\c{c}\~ao de cat\'alogos e menus
    \item Monitoriza\c{c}\~ao de pedidos e relatorios
\end{itemize}

\section{Conclus\~ao}

O projeto desenvolvido demonstra como uma arquitetura bem planeada facilita a gest\~ao de fluxos operacionais complexos. Ao separar a interface da l\'ogica e esta dos dados atrav\'es de fachadas independentes e DAOs especializados, conseguimos criar um sistema onde a rapidez no atendimento \'e suportada por uma organiza\c{c}\~ao eficiente da produ\c{c}\~ao.

A estrutura em tr\^es subsistemas especializados, cada um com a sua pr\'opria fachada e responsabilidades bem definidas, garante que o sistema est\'a preparado para ser utilizado em ambientes reais, oferecendo uma base est\'avel para o crescimento da cadeia de restaurantes. O padr\~ao de DAOs para persist\^encia baseado em SQL assegura a escalabilidade e a facilidade de manuten\c{c}\~ao futura.

\end{document}
